
\chapter{Werkbeschreibung}
%\label{chap:werkbeschreibung}

\begin{table}[hbt]
  \begin{tabular}{l|cc}
    Werkbezeichnung: &  \\
    \hline
    Titel: & & \\
    \hline
    Serie: & \\
    \hline
    Künstler*In: & \\
    \hline
    Datierung: & \\
    \hline
    Masse: & Höhe: cm, Breite: cm, Tiefe: cm \\
    \hline
    Material / Technik: & \\
   
  \end{tabular}
  \caption{}
\end{table}


Projektnummer, Werkbezeichnung: Titel/Darstellung, Künstler*In, Datierung, Masse, Gewicht, Material/Technik, Zugehörige Teile, Ältere Restaurierungen, Werkbeschreibung 

\section{Kunst- und kulturgeschichtliche Einordnung}
\subsection{Darstellung bzw. Beschreibung}
\subsection{Künstler*In, Datierung, Provenienz}
\subsection{Allgemeine Bemerkungen zum Werk}

\section{Technischer Befund}
\subsection{Technischer Aufbau und Konstruktion}
\subsection{Materialität}
\subsection{Funktionalität}





