% This is part of the document "Project documentation template".
%https://intranet.bfh.ch/BFH/de/Dienste/Kommunikation/CI/Corporate-Design/Downloads/Office-Vorlagen/Seiten/default.aspx
% Authors: brd3, kaa1
% Angepasst auf KuR-Layout von Ralph Michel / 27.05.2019
% Angepasst auf Masterarbeit-Vorlage von Elisa Carl / 2020
% Angepasst auf neue KuR-Vorlage von Ralph Michel / 22.10.2021


% Hauptdokument

%========================================

\documentclass[
%draft,
	a4paper, 				% Papierformat
	9.5pt,						% Schriftgr�sse
%	twoside,					% Zwei-Seitig 
%	openright,			 	% Neues Kaptiel auf rechter Seite
	notitlepage,		 	% Keine Standard-Titelseite
	parskip=half,	  		 % set paragraph skip to half of a line
	listof=entryprefix, 	% Pr�fix (Abb.)in den Verzeichnissen
]{scrreprt}			    	% KOMA-script Klasse

%------------------------------------------


%\raggedbottom ----f�r report schon voreingestellt
\renewcommand*{\chapterheadstartvskip}{\vspace*{-\topskip}} 		% Abstand auf neuer Seite reduzieren
%\KOMA-options{cleardoublepage=plain}												% Kopf- und Fusszeile auf leerer Seite


% ------------Hyperref Package (Links in PDF)-----------------------

\usepackage[
pdftex,ngerman,bookmarks,plainpages=false,pdfpagelabels,
backref = {false},											% No index backreference
colorlinks = {true},               			   				% Color links in a PDF
hypertexnames = {true},              					% no failures "same page(i)"
bookmarksopen = {true},               				% opens the bar on the left side
bookmarksopenlevel = {0},             				% depth of opened bookmarks
pdftitle = {PDF Titel},	   									%TODO PDF Titel setzen 
pdfauthor = {Autor},        					 			%TODO PDF Autor definieren 
pdfsubject = {LaTeX Vorlage},        			   	%TODO PDF Inhalt definieren
linkcolor = {linkcolor},              						% Color of Links
citecolor = {linkcolor},              						% Color of Cite-Links
urlcolor = {linkcolor},               						% Color of URLs
]{hyperref}


% ------------Schrift----------------

% ----------Verzeichnistiefe------------
\usepackage{tocvsec2}						   % Kontrolle �ber Verzeichnistiefe
\setcounter{tocdepth}{3}						% Verzeichnistiefe im Inhaltsverzeichnis definieren
\setcounter{secnumdepth}{3} 				% Verzeichnistiefe im Text

\RedeclareSectionCommand[
tocbeforeskip = 7pt]{chapter} 				%Abst�nde im Inhaltsverzeichnis

%--------Formatierung der �berschriften------------------------

\addtokomafont{chapter}{\Large\normalfont} 
\addtokomafont{section}{\normalsize} 
\addtokomafont{subsection}{\normalsize\normalfont} 
\addtokomafont{subsubsection}{\normalsize\normalfont} 
\usepackage{microtype}

% ------Abst�nde �berschriften definieren-----------------------

\RedeclareSectionCommand[
beforeskip=18bp,
afterskip=6bp]{chapter}

\RedeclareSectionCommand[
beforeskip=12bp,
afterskip=2bp]{section}

\RedeclareSectionCommand[ 
beforeskip=6bp,
afterskip=0.3bp]{subsection}

\RedeclareSectionCommand[ 
beforeskip=6bp,
afterskip=0.3bp]{subsubsection}

%---Sprache-----------

\usepackage[ngerman]{babel}				   	% Deutsche Interpunktion
\usepackage[latin1]{inputenc}			   		% Unix/Linux - load extended character set (ISO 8859-1)
%\usepackage[ansinew]{inputenc} 		% Windows - load extended character set (ISO 8859-1)\
\usepackage[T1]{fontenc}					  	% Silbentrennung mit W�rter � � �
\usepackage[autostyle=true,german=quotes]{csquotes} % deutsche Anf�hrungszeichen
\usepackage{textcomp}						 	% Zus�tzliche Symbole
\usepackage{ae}									   	% Bessere Aufl�sung f�r T1-Schriften
\usepackage[scaled=0.92]{helvet}  		% ??????

\renewcommand*\familydefault{\sfdefault} 			   % Sans Serif familiy ausgew�hlt


%-------Zus�tze----
\usepackage{fancyhdr}						   	% Einfache Anpassung f�r Kopf- und Fusszeile
\usepackage{etoolbox}						   	% Farbanpassung f�r Kopf- und Fusszeile
\usepackage{graphicx}						    % Einbinden von Bilder
\usepackage{float}							    % Exakte Positionierung von Bildern
\usepackage[format=plain,                  % Captions ausrichten und formartieren 
justification=justified,                			% linksb�ndig ausrichten 
singlelinecheck=false,          					% Einzeilige Captions nicht zentrieren 
font=footnotesize,                                 % Kleine Schriftart verwenden 
%labelfont=bf,                                       % Abbildung fett drucken 
position=top]                       					% Abst�nde f�r �berschriften verwenden 
{caption}						     						% Bild- und Tabellenunterschriften


\usepackage{blindtext}

\usepackage{chngcntr}							% Fortlaufende Nummerierung
\counterwithout{figure}{chapter} 
\counterwithout{table}{chapter}
\renewcaptionname{ngerman}{\figurename}{Abb.} 
\renewcaptionname{ngerman}{\tablename}{Tab.}

\KOMAoption{listof}{entryprefix} 			% Pefix Abb. im Verzeichnis

\usepackage{booktabs}						% Paket f�r sch�nere Tabellen
\usepackage{setspace} 						% Zeilenabstand





%-------------Mathe Pakete--------------------------------

\usepackage{amsmath}                    	   	% various features to facilitate writing math formulas
\usepackage{amsthm}                       	 	% enhanced version of latex's newtheorem
\usepackage{amsfonts}                      		% set of miscellaneous TeX fonts that augment the standard CM
\usepackage{amssymb}							% mathematical special characters
\usepackage{exscale}

%-------------Chemie Pakete-------------------


\usepackage[version=4]{mhchem}
\usepackage{chemfig}

\newcommand\setpolymerdelim[2]{\def\delimleft{#1}\def\delimright{#2}} \def\makebraces[#1,#2]#3#4#5{%
	\edef\delimhalfdim{\the\dimexpr(#1+#2)/2}% 
	\edef\delimvshift{\the\dimexpr(#1-#2)/2}%
	\chemmove{%
		\node[at=(#4),yshift=(\delimvshift)]
		{$\left\delimleft\vrule height\delimhalfdim depth\delimhalfdim width0pt\right.$};%
		\node[at=(#5),yshift=(\delimvshift)]
		{$\left.\vrule height\delimhalfdim depth\delimhalfdim width0pt\right\delimright_{\rlap{$\scriptstyle#3$}}$};}} 


%-------------Zitier und Bibliographie Paket-----------------------

\usepackage[backend=biber,style=authoryear,firstinits=true,maxcitenames=1,autocite=inline,isbn=false]{biblatex}
\addbibresource{datenbanken/bibliography.bib}				%TODO Bibliographie Quelle angeben 
\setlength{\bibitemsep}{1em}      								% Abstand zwischen den Literaturangaben

%\urlstyle{rm}

%-------------Pakete zum erleichterten platzieren von Objekten-----------------
		
\usepackage[absolute]{textpos} 			% f�r command textblock, genaue Positionierung von Textabschnitten
\setlength{\TPHorizModule}{1mm}
\setlength{\TPVertModule}{1mm}

\usepackage{pdfpages} 						% PDF Datei einf�gen
\includepdfset{offset=0.7cm 0cm} 		%Positionierung pdf damit im Textblock

%-------------Farben definieren-----------------------------------

\RequirePackage{color}											% Color (not xcolor) 
\definecolor{linkblue}{rgb}{0,0,0.8} 					% Standard
\definecolor{darkblue}{rgb}{0,0.08,0.45} 			% Dunkelblau
\definecolor{bfhgrey}{rgb}{0.41,0.49,0.57}			% BFH Grau
%\definecolor{linkcolor}{rgb}{0,0,0.8}  				% Blau f�r Web und CD-Version
\definecolor{linkcolor}{rgb}{0,0,0}   					% Schwarz f�r die Druckversion

%-------------Seite definieren----------------------------------------

\usepackage{geometry}
\geometry{
	a4paper,
	left=30mm,
	right=20mm,
	top=22mm,
	%headheight=20mm,
	%headsep=10mm,
	textheight=260mm,
	footskip=6mm
}


%-----------------Makeindex Paket---------------------------

\usepackage{makeidx}				% Index erzeugen
\makeindex								 	% Index initialisieren
%\addtocontents{toc}{\vspace{-2ex}}

%\usepackage{tocloft}
%\cftbeforechapskip
%\cftsetindents{chapter}{em}{1em}
%\cftsetindents{section}{0em}{2em}
%\cftsetindents{subsection}{0em}{2em}
%\newtocstyle[KOMAlike]{compressed}{
%	\settocfeature[0]{entryvskip}{0.5em plus 1pt}%  % Die Stellschraube


%================= Intro ============================

\begin{document}									% Start Dokument

%\pagenumbering{roman}				  % Stil der Seitenzahlnummerierung

%----------------Vorspann-------------------------------------

\providecommand{\titel}{Dokumentation}		%  TODO Hier den Titel des Berichts/Thesis eingeben					% TODO Titel der Arbeit definieren - Titel der Arbeit aus Datei titel.tex lesen
%  Hier die aktuelle Versionsnummer eingeben
\providecommand{\versionnumber}{0.1}	

%  Hier das Datum der aktuellen Version eingeben
\providecommand{\versiondate}{21.07.2022}					% TODO Versionsnummer nachtragen - Versionsnummer und Datum aus Datei version.tex lesen ...?

%----------------Kopf- und Fusszeile---------------------------

\makeatletter
\patchcmd{\@fancyhead}{\rlap}{\color{bfhgrey}\rlap}{}{}			% Neue Farbe f�r Kopfzeile
\patchcmd{\@fancyfoot}{\rlap}{\color{bfhgrey}\rlap}{}{}			% Neue Farbe f�r Fusszeile
\makeatother

\fancyhf{}																						% Alle Felder l�schen
\fancypagestyle{plain}{																	% Neue Definition von plain style (?)
	\fancyfoot[OR,EL]{\color{bfhgrey}\footnotesize \thepage}	% Fusszeile rechts -> Seitenzahl
	\fancyfoot[OL,ER]{\color{bfhgrey}\footnotesize Titel der Doku | Vorname Name | MMM | Version \versionnumber -  \versiondate}																% TODO Fusszeile links definieren
} 


\renewcommand{\chaptermark}[1]{\markboth{\thechapter.  #1}{}}			% (?)
\renewcommand{\headrulewidth}{0pt}														% Keine Linie Kopf
\renewcommand{\footrulewidth}{0pt} 														% Keine Linie Fuss	




%------------Titelseite und Abstract--------------------------------


%\include{vorspann/titelseite_ohne_bild}				% Aktivieren f�r Titelseite ohne Bild
% Titelseite mit Bild

\begin{titlepage}


% BFH-Logo absolute placed at (28,12) on A4 and picture (16:9 or 15cm x 8.5cm)
% Actually not a realy satisfactory solution but working.
%---------------------------------------------------------------------------
\setlength{\unitlength}{1mm}
\begin{textblock}{20}[0,0](180,260)
	\includegraphics[scale=0.7]{bilder/BFH_Logo.png}
\end{textblock}

\setlength{\unitlength}{1mm}
\begin{textblock}{20}[0,0](28,22)
	\includegraphics[scale=0.25]{bilder/HKB_Logo.png}
\end{textblock}

\setlength{\unitlength}{1mm}
\begin{textblock}{20}[0,0](80,22)
	\includegraphics[scale=0.65]{bilder/Swiss_Conservation_Campus.jpg}
\end{textblock}


\begin{textblock}{154}[0,0](28,90)
	\includegraphics[height=11cm,width=10cm]{bilder/titelseite/testbild.jpg} 	% TODO Titelbild definieren
\end{textblock}



% TODO Institution / Titel / Untertitel / Autoren / Experten:
%---------------------------------------------------------------------------
\begin{flushleft}

\vspace*{30mm}

\fontsize{26pt}{28pt}\selectfont 
\textbf{\titel}			\\							% Titel aus der Datei vorspann/titel.tex lesen
\vspace{2mm}

\fontsize{16pt}{20pt}\selectfont\vspace{0.3em}
Untersuchung, Konservierung, Restaurierung		\\							% TODO Untertitel eingeben
\vspace{5mm}





\begin{textblock}{150}(28,200)
\fontsize{10pt}{12pt}\selectfont
\begin{tabbing}
xxxxxxxxxxxxxxxxxxxxxxxx\=xxxxxxxxxxxxxxxxxxxxxxxxxxxxxxxxxxxxxxxxxxxxxxx \kill
Projektnummer: 				\>           \\
Werkbezeichnung: \>          \\
Titel / Darstellung: \>             \\
KünstlerIn: \>                       \\
Datierung: \>						\\
Masse: \> Höhe: Breite: Tiefe  				\\
Material / Technik: \>								\\
Auflage / Serie: \>										\\
Problemstellung / Auftrag: \>						\\
AuftraggeberIn: \>										\\
Inventar-Nr.: \>											\\
EigentümerIn / Standort: \>									\\
Bearbeitet von: \>													\\
Verantwortlich: \>												\\
Datum: \>														\\

\end{tabbing}

\end{textblock}
\end{flushleft}

%\begin{textblock}{150}(28,280)
%\noindent 
%\color{bfhgrey}\fontsize{9pt}{10pt}\selectfont
%Berner Fachhochschule | Haute école spécialisée bernoise | Bern University of Applied Sciences
%\color{black}\selectfont
%\end{textblock}


\end{titlepage}
					% TODO Titelseite mit Bild definieren - Aktivieren f�r Titelseite mit Bild
\pagestyle{plain}														% Layout mit Seitennummerierung/nur fusszeile keine kopfzeile wird gedruckt
%  Hier die aktuelle Versionsnummer eingeben
\providecommand{\versionnumber}{0.1}	

%  Hier das Datum der aktuellen Version eingeben
\providecommand{\versiondate}{21.07.2022}										   		% Aktivieren f�r Vorspann Versionen
%\cleardoubleemptypage											% (?)
%\setcounter{page}{1}											    % Start bei Seitenzahl 1
%\cleardoublepage													% (?)
%\phantomsection 													% (?)
%\cleardoubleemptypage											% (?)


%--------------Vorspann-------------------------------------

\pagenumbering{arabic}											% Stil der Seitenzahl
\setcounter{page}{2}											      	% Start bei Seitenzahl 1												
%\chapter*{Zusammenfassung/ Abstract}
\addcontentsline{toc}{chapter}{Zusammenfassung/ Abstract}
\newpage



\chapter*{Dank}
\addcontentsline{toc}{chapter}{Dank}											% TODO Abstract  und Dank JA/NEIN
%\chapter*{Abk�rzungsverzeichnis}
\addcontentsline{toc}{chapter}{Abk�rzungsverzeichnis}
\newpage

\chapter*{Glossar}
\addcontentsline{toc}{chapter}{Glossar}				% TODO Abk�rzungsverzeichnis und Glossar JA/NEIN
\tableofcontents															% Inhaltsverzeichnis
%\cleardoublepage?

%----------------Hauptteil---------------------------------------------



\chapter{Vorbemerkungen}
\label{chap:vorbemerkungen}

Abkürzungsverzeichnis, Bemerkungen zur Dokumentationsform oder allgemeine Informationen: z.B.: 

\begin{itemize}
	\item wer hat das Werk vorher und bis wohin bearbeitet
	\item was waren/sind die Hauptprobleme, evtl. muss die Verpackung dokumentiert werden
	\item Sichtweise (z.B. Gemälde vom Betrachter aus, Skulptur von der Skulptur aus, Architektur (Himmelsrichtungen)
\end{itemize}


\section{LaTeX Beispiele}

Quisque tristique venenatis\footnote{Dies ist eine Fussnote} orci euismod eleifend. Curabitur vitae consequat diam. Fusce non nisi pellentesque dui venenatis laoreet (siehe Abbildung \ref{fig:testbild}) Phasellus eleifend lobortis lobortis. Aenean mattis at orci ut volutpat. \autocite[S. 55]{laurencew.mckeenEffectUVLight2019}


Praesent volutpat magna nec sollicitudin malesuada. Praesent quis dignissim nisi. Cras ac vulputate urna. Phasellus magna arcu, sagittis non lacus eu, semper lacinia nisl. Etiam quis pellentesque tortor, eu tempor magna. Nunc a ex vitae diam blandit maximus at non leo. Etiam tempor rutrum dui. Proin vestibulum ex id velit suscipit pellentesque. Pellentesque dictum urna eu dolor mattis cursus.




\bigskip

\begin{figure}[h]
	\centering
	\captionsetup{margin=2.4cm} % caption verschoben
	\includegraphics[width=0.7
	\linewidth]{bilder/identifikation/testbild}
	\caption{Bildunterschrift (Fotograf)}
	\label{fig:testbild}
\end{figure}


\begin{figure}[H]
	\centering
	\begin{minipage}[t]{0.48\textwidth}
		\includegraphics[width=\textwidth]{bilder/identifikation/testbild}
		\caption{Bildunterschrift}
		\label{fig:}
	\end{minipage}
	\hfill
	\begin{minipage}[t]{0.48\textwidth}
		\includegraphics[width=\textwidth]{bilder/identifikation/testbild}
		\caption{Bidunterschrift test Bidunterschrift test Bidunterschrift test Bidunterschrift test }
		\label{fig:}
	\end{minipage}
\end{figure}


























\chapter{Identifikation}
\label{chap:identifikation}































\chapter{Werkbeschreibung}
%\label{chap:werkbeschreibung}

\begin{table}[hbt]
  \begin{tabular}{l|cc}
    Werkbezeichnung: &  \\
    \hline
    Titel: & & \\
    \hline
    Serie: & \\
    \hline
    K�nstler*In: & \\
    \hline
    Datierung: & \\
    \hline
    Masse: & H�he: cm, Breite: cm, Tiefe: cm \\
    \hline
    Material / Technik: & \\
   
  \end{tabular}
  \caption{}
\end{table}


Projektnummer, Werkbezeichnung: Titel/Darstellung, K�nstler*In, Datierung, Masse, Gewicht, Material/Technik, Zugeh�rige Teile, �ltere Restaurierungen, Werkbeschreibung 

\section{Kunst- und kulturgeschichtliche Einordnung}
\subsection{Darstellung bzw. Beschreibung}
\subsection{K�nstler*In, Datierung, Provenienz}
\subsection{Allgemeine Bemerkungen zum Werk}

\section{Technischer Befund}
\subsection{Technischer Aufbau und Konstruktion}
\subsection{Materialit�t}
\subsection{Funktionalit�t}






\chapter{Zustandsbeschreibung}
%\label{chap:zustandbeschreibung}

Optische Untersuchung, Auswertung Apparative Untersuchung, Restaurierungen/Reparaturen, Kartierungen, etc.





\chapter{Schadensursachen}
%\label{chap:schadensursachen}


Beschreibung von Ursachen, Anl�ssen, Zusammenh�ngen
\include{kapitel/konzeptkonservierung}
\chapter{Durchgeführte Massnahmen}
%\label{chap:durchgefuertemassnahmen}


\include{kapitel/verpackung}




%----------------Anhang--------------------------------------

\chapter{Verzeichnisse}

\section{Literaturverzeichnis}

\printbibliography[heading=none]

\section{Materialverzeichnis}

\section{Ger�rteverzeichnis}

\section{Abbildungsverzeichnis}

\begingroup
\let\clearpage\relax % Unterdr�ckt Seitenumbruch
\renewcommand{\listfigurename}{} % Unterdr�ckt �berschrift
\vspace{-14pt}
\renewcommand*{\addvspace}[1]{}% Verhindert gr�sseren abstand zwischen Kapiteln
\listoffigures
\addtocontents{lof}{\protect\vspace{10pt}}% Verhindert gr�sseren abstand zwischen Kapiteln
\addtocontents{lof}{\protect\renewcommand*\protect\addvspace[1]{}}% Verhindert gr�sseren abstand zwischen Kapiteln
\endgroup


\section{Tabellenverzeichnis}
\begingroup
\let\clearpage\relax
\renewcommand{\listfigurename}{} 
\renewcommand{\listtablename}{} 
\vspace{-14pt}
\listoftables
\addtocontents{lof}{\protect\vspace{10pt}}% Verhindert gr�sseren abstand zwischen Kapiteln
\addtocontents{lof}{\protect\renewcommand*\protect\addvspace[1]{}}% Verhindert gr�sseren abstand zwischen Kapiteln
\endgroup

\include{anhang/beispielanhang}
\chapter{Selbstständigkeitserklärung und Nutzungsrecht}
%\label{chap:schadensursachen}

Ich versichere, dass ich die vorliegende Arbeit selbstständig und ohne Benutzung anderer als der im Literaturverzeichnis angegebenen Quellen und Hilfsmittel angefertigt habe. Sämtliche Textstellen und Materialien wie z.B. Abbildungen, Grafiken, Tabellen etc., die nicht von mir stammen, sind als Zitate gekennzeichnet und mit dem genauen Hinweis auf ihre Herkunft versehen.

Ich übertrage der Hochschule der Künste Bern HKB und dem Fachbereich Konservierung und Restaurierung das einfache Nutzungsrecht, um Kopien der Arbeit herzustellen und zu verbreiten.\\



Ort, Datum: \hspace*{35mm} Unterschrift:













%------------Literaturverzeichnis----------------------------------------

%\cleardoublepage
%\phantomsection 

%\addcontentsline{toc}{chapter}{Literaturverzeichnis}
%\renewcommand{\refname}{Literaturverzeichnis}
%\printbibliography 


%------------Abbildungs- und Tabelleverzeichnis--------------------------------------

%\cleardoublepage?
%\phantomsection 
%\addcontentsline{toc}{chapter}{Abbildungsverzeichnis}
%\listoffigures
%cleardoublepage?

%\phantomsection 
%\addcontentsline{toc}{chapter}{Tabellenverzeichnis}
%\listoftables


%--------------Anhang----------------------------------------------------------------

%\appendix
%\settocdepth{subsection} ?



\end{document}					% Ende des Dokuments
